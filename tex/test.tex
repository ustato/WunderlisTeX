\documentclass[a4j]{jsarticle}

% Unicode文字を使用できるようにする
\usepackage[T1]{fontenc}
\usepackage[utf8]{inputenc}

% 数式の記述に用いる
\usepackage{amsmath}

% PythonTeXを利用する
\usepackage[]{pythontex}

% Pythonの関数をLaTeXのコマンドで呼び出せるようにする
% 関数は以下で定義
\newcommand{\iseven}[1]{\py{iseven(#1)}}

\title{Python \TeX のテスト}
\author{はむ吉(のんびり)}
\date{\today}

\begin{document}
\maketitle
\section{Hello, Python \TeX !}
\begin{pycode}
#!/usr/bin/env python2
import yaml
import json
from requests_oauthlib import OAuth2Session

# API tokenを'secret.yaml'から参照する
yaml_dict = yaml.load(open('../secret.yaml'), Loader=yaml.SafeLoader)
client_id = yaml_dict['client_id']
access_token = yaml_dict['access_token']

url = "https://a.wunderlist.com/api/v1/lists"
params = {}

wunderlist = OAuth2Session()
wunderlist.headers['X-Client-ID'] = client_id
wunderlist.headers['X-Access-Token'] = access_token
req = wunderlist.get(url, params=params)

if req.status_code == 200:
    lists = json.loads(req.text)

    for list_ in lists:
        url = "https://a.wunderlist.com/api/v1/tasks"
        params = {
                "list_id": list_["id"],
                "completed": True
                }
        done_tasks = json.loads(wunderlist.get(url, params=params).text)

        params["completed"] = False
        req = wunderlist.get(url, params=params)
        notdone_tasks = json.loads(wunderlist.get(url, params=params).text)

        tasks = done_tasks + notdone_tasks

        n_tasks = len(tasks)
        n_dones = len([task for task in tasks if task["completed"]==True])
        if (n_tasks != 0):
            print("{0} ({1:.2f} \%)".format(list_["title"], n_dones / n_tasks))

        for task in tasks:
            print("\\begin{itemize}")
            if (task["completed"]):
                print("\\item - {}: done".format(task["title"]))
            else:
                print("\\item - {}: not done".format(task["title"]))
            print("\\end{itemize}")
else:
    print("Error: \%d" % req.status_code)
\end{pycode}

\section{Python側で定義した関数を\LaTeX 側から使う}
\begin{pycode}
def iseven(n):
    if type(n) is not int:
        return u"{0}は整数ではありません。".format(n)
    elif n % 2 == 0:
        return u"{0}は偶数です。".format(n)
    elif n % 2 == 1:
        return u"{0}は奇数です。".format(n)
\end{pycode}
\begin{itemize}
\item \iseven{1024}
\item \iseven{59049}
\item \iseven{8.314}
\end{itemize}

\section{SymPyを利用する}

\begin{sympycode}
x = symbols("x")
func = x**4 + x**3 + x**2 + x + 1

print(r"\begin{align}")
for n in range(1, 5):
    deriv = Derivative(func, x, n)
    print("{0} &= {1}".format(latex(deriv), latex(deriv.doit())))
    if n < 4:
        print(r"\\")
print(r"\end{align}")
\end{sympycode}
\end{document}
