\documentclass[a4j]{jsarticle}

% Unicode文字を使用できるようにする
\usepackage[T1]{fontenc}
\usepackage[utf8]{inputenc}
\usepackage{amsmath}
\usepackage{txfonts}

% チェックボックスに用いる
\usepackage{bbding}
\usepackage{pifont}
\usepackage{wasysym}
\usepackage{amssymb}

% プログレスバーに用いる
\usepackage[dvipdfmx]{graphicx}
\usepackage{calc}
\usepackage{kvsetkeys}
\usepackage{kvoptions}
\usepackage{tikz}
\usepackage{progressbar}

% PythonTeXを利用する
\usepackage{additional}
\usepackage{pythontex}

% Pythonの関数をLaTeXのコマンドで呼び出せるようにする
% 関数は以下で定義
\newcommand{\todotable}[1]{\py{todolist.get_todotable()}}
\newcommand{\doneitem}[1]{\py{todolist.get_completed_task_in_a_week()}}


\title{打ち合わせ資料}
\author{あざ◎}
\date{\today}

\begin{document}
\maketitle

% Pythonコードを埋め込む
\begin{pycode}
from todolist import ToDoList
todolist = ToDoList()
\end{pycode}

\section{前回までの打ち合わせ内容}
\begin{itemize}
	\item ``多重トピックテキストの確率モデル(1)'' を読んだ
	\item ナイーブベイズをラグランジュの未定乗数法で解く(理解してない)
	\item 抄録提出(実験はしていない)
	\item 8月はあってないような短さなので,実験計画を立てる
\end{itemize}

\section{今週までにやったこと}
\doneitem

\section{来週やること}
\begin{itemize}
	\item 防災コンペの精度向上と発表に使えるようにしたい(実験必須)
	\item LDAを勉強する(引き続き)
	\item ゼミはLDA関連をやる
\end{itemize}

\clearpage
\section{やること,やったこと}
\todotable

\end{document}
